\documentclass[main.tex]{subfiles}

\begin{document}
\chapter{Planare dieelektrische Wellenleiter}
\begin{figure}[h!]
    \centering
    \subfile{imgs/dielektrischeWellenleiter.tex}
    \caption{Dielektrischer Wellenleiter mit drei unterschiedlichen (generalisierten) Permittivitäten.}
    \label{fig:skizze1}
\end{figure}

In den nachfolgenden Herleitungen nehmen wir Translationsinvarianz in $y$-Richtung an und die Wellen sollen in $z$-Richtung propagieren. Wegen der Translationsinvarianz verschwindet die Ableitung $∂_y H_{z,i}^p$ der Feldkomponente des Modenprofils im Bereich $i$ in der transversalen Helmholtzgleichung. \\
Werden transversal-elektrische Wellen betrachtet, steht also das elektrische Feld $\vec E$ stets senkrecht zur Ausbreitungsrichtung $\ez$ und es gilt $E_z = 0$, untersuchen wir zunächst die $z$-Komponente der $H$-Moden. Schreiben wir wie üblich
$$\vec H = \vec H^p e^{-jk_z z},$$
so gilt dann in den drei Bereichen
\begin{equation}
    \label{eqn:helmholtz}
    ∂_x^2 H_{z,i}^p + k_{c,i}^p H_{z,i}^p = 0 \qquad\text{für $i=1,2,3.$}
\end{equation}
Dabei muss die Dispersionsrelation für den Wellenvektor erfüllt sein:
\begin{equation}
    \label{eqn:dispersion}
    k_{c,i}^2 = ω^2 μ_i ε_i - k_{z,i}^2 = \frac{ω^2 n_i^2}{c_0^2} - β^2 \qquad\text{für $i=1,2,3$.}
\end{equation}
Aus dem Produktansatz erhält man wegen $∂_y H_{z,i}^p = 0$ auch $k_{y,i} = 0$ und somit $k_{c,i}^2 = k_{x,i}^2$.
Die Gleichung \eqref{eqn:helmholtz} hat bekannte Fundamentallösungen $\exp \left( \pm jk_{x,i} x\right), \sin(k_{x,i} x)$ oder $\cos(k_{x,i},x)$. Da wir an geführten Moden interessiert sind, muss im Mantel, also für die Regionen $i=2,3,$ mit $|x|\ge h/2$, das Feld abklingen, also gilt dort
$$H_{z,i}^p ∝ e^{-γ_i |x|}, \qquad i=2,3,$$
mit positivem, reeller Abklingkonstante $γ_i = jk_{x,i} = \sqrt{β^2 - \frac{ω n_i^2}{c_0^2}}$, und also $β^2 > \frac{ω^2 n_i^2}{c_0^2}.$\\
Im Kern hingegen soll $k_{1,x}$ reell sein, also $β\le \frac{ωn_1}{c_0}.$ Das heißt folglich, dass geführte Moden möglich sind, wenn die Propagationskonstante im Intervall $β \in \left[ \frac{ω}{c_0} n_{2,3}, \frac{ω}{c_0} n_1 \right]$ ist.

Der Einfachheit halber betrachte nun einen symmetrischen Filmwellenleiter mit $n_2 = n_3$ und $γ:= γ_2 = γ_3 = jk_{x,2}$. Man unterscheidet zwischen ungeraden und geraden Moden in dem Sinne, ob die $y$-Komponente von $E$ eine ungerade oder gerade Funktion ist.

\section{Ungerade Moden}
Aus den vorigen Überlegungen haben wir
\begin{equation}
    \label{eqn:magnetisch}
    H_z^p = \begin{cases}
    a \exp \left( γ x \right)&\text{für $x \le \frac{-h}{2}$},\\
    b \cos (k_{1,x} x ) &\text{für $|x| \le \frac{h}{2}$},\\
    a \exp \left( γ x \right)&\text{für $x \ge \frac{h}{2}$}.
    \end{cases}
\end{equation}
Dann ergibt sich der transversale modale Feldanteil
$\vec E ^T = \begin{pmatrix}
E_x^P\\
E_y^P\end{pmatrix}$
aus den Maxwell-Gleichungen zu
\begin{equation}
    \label{eqn:eletrischTransversal}
    \begin{aligned}
    \vec E_i^T &= \frac{jωμ_0}{k_{i,x}^2} \ez \times \overbrace{\grad_T}^{\ex ∂_x} H_{z,i}^P - \frac{jβ}{k_{i,x}^2} \overbrace{E_z^P}^{=0}\\
    &= jωμ_0\ey \cdot \begin{cases}
    a γ \exp \left( γx \right) \frac{1}{-γ^2}&\text{für $x \le \frac{-h}{2}$},\\
    -b k_{1,x}\sin(k_{1,x} x) \frac{1}{k_{1,x}^2} &\text{für $|x| \le \frac{h}{2}$},\\
    -aγ \exp(-γx) \frac{1}{-γ^2} &\text{für $x \ge \frac{h}{2}$}.
    \end{cases}
    \end{aligned}
\end{equation}
Im die Koeffizienten $a$ und $b$ zu bestimmen nutzt man die Stetigkeit der Tangentialkomponenten von $\vec E $ und $\vec H$ an den Grenzübergängen. 
Mit \eqref{eqn:eletrischTransversal} folgt bei $x = \frac{h}{2}$:
\begin{equation}
    \label{eqn:bed1}
    E_{1,y}^p(h/2) = E_{z,y}^P(h/2) \Rightarrow \frac{a}{γ} \exp\left( \frac{-γh}{2} \right) = - \frac{b}{k_{1,x}}\sin \left(k_{1,x} \frac{h}{2}\right)
\end{equation}
Analog liefert \eqref{eqn:magnetisch}
\begin{equation}
    \label{eqn:bed2}
    a \exp\left( \frac{-γh}{2}\right) = b\cos \left( k_{1,x} \frac{h}{2}\right) 
\end{equation}
Aus \eqref{eqn:bed1} und \eqref{eqn:bed2} erhält man eine transzendente Bestimmungsgleichung für $k_{1,x}$:
$$-\frac{k_{1,x}}{γ} = \tan \left( \frac{k_{1,x }h}{2}\right).$$
\end{document}